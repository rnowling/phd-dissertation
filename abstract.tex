\begin{abstract}
  Bioinformaticians are constantly developing new algorithms, searching for more accurate answers, more quickly.  Implementation of novel algorithms is incredibly time and resource intensive, however.  Challenges from debugging complex algorithms are compounded by the use of low-level languages and parallel processing to achieve optimal performance and scale to large data sets.

  High-quality machine learning libraries with high-performance, parallel implementations of common algorithms and easy-to-use APIs in high-level languages have become widely available. By utilizing existing machine learning libraries, bioinformatics developers can reduce development time and labor.  

  We demonstrate how traditional domain-specific bioinformatics methods can be improved and replaced by machine learning techniques through studies of two problems (gene annotation and population genetics) arising from insect vector genomics.
\end{abstract}
