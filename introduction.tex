\chapter{\uppercase{Introduction}}

Arthropod vectors are increasingly becoming important in emerging human disease outbreaks.  In addition to the recent Zika outbreak vectored by the mosquito \emph{Aedes aegypti}, linked to serious medical outcomes, nearly half of the world's population is still at risk of contracting malaria.  In 2015, 214 million malaria infections, resulting in 438,000 deaths, were reported. Nearly 90\% of the reported infections and deaths have occurred in Sub-Saharan Africa.  Mosquitoes in the \emph{Anopheles} genus are the sole vectors of the malaria-causing \emph{Plasmodium} parasites \cite{Neafsey2015,Neafsey2010,Lawniczak2010}.

After malaria, visceral leishmaniasis is the second-largest parasitic killer in the world, responsible for 200,000 to 400,000 infections per year worldwide.  Like malaria, leishmaniasis is vectored by insects, in particular phlebotomine sand flies such as \emph{Phlebotomus papatasi} and \emph{Lutzomyia longipalpis}.

Efforts to control populations of insect vectors such as \emph{Anopheles gambiae} are key to efforts to eradicate malaria, leishmaniasis, and other insect-borne diseases \cite{Holt2002}.  \emph{Anopheles} species have been subject to evolutionary adapations due to host-parasite interactions and specializations needed to thrive when living among and feed on humans \cite{Neafsey2015}. Population control efforts are informed through studies of insect biology and characterizerations of these adaptations, which are enabled by sequencing and comparative analysis of insect genomes.   Such studies have been fruitful, with results including the identification of chemosensory receptors responsible for detection of human hosts and the molecular bases of variations in insecticide resistance between populations and species \cite{Lawniczak2010}.

Bioinformatic analysis of insect genomes is complicated by unique characteristics of insect biology.  For example, mammalian chemosensory receptors are G Protein Coupled Receptors (GPCRs), while insect chemosensory receptors are ion channels which are poorly-conserved on the sequence level \cite{Sato2008,Touhara2009,Wicher2008}.  Insect genomes tend to have low linkage disequilibrium (LD), or correlation  between variations located nearby on chromosomes, resulting in low statistical power when identifying interesting variations.

The larger bioinformatics community is adept at developing algorithms and accompanying software to meet such challenges.  Developing methods and software is expensive in terms of time and resources (human, financial), however. A majority of bioinformatics software is developed for a single or small number of analyses, abandoned once the analyses are done. Studies of defect rates in software have found that bugs tend to most frequently occur in files with the most recent activity \cite{Ostrand2005}.  Interpreted another way, new bugs can be introduced in the process of adding new features and fixing previously-known bugs. Over time, however, modules mature and the rate at which new bugs are found declines, leading to reliable components. Furthermore, bioinformatics applications often employ specalized methods not available in common libraries.  As such, bioinformatics software often consists of completely custom implementations of every component from file parsers to complex algorithms and statistical tests.  We can conclude that most bioinformatics applications consist of new, custom code with high defect rates, leading to significant development costs due to time spent debugging.

Furthermore, bioinformatics software is often used on large data sets and employs computationally-expensive algorithms.  Since reducing software run times from hours or days to minutes can significantly boost productivity, enabling a much shorter loop in idea-implementation-evaluation cycles, it is highly desireable to optimize and parallelize code.  Low-level languages like C are often used to achieve optimal usage of hardware. C is more error-prone than high-level langauges like Python, however, due to manual memory management and poor type safety.  Parallelization is also known to increase software complexity and defect rates, especially since race conditions are notoriously difficult to debug. Both requirements lead to increased time and effort spent debugging and validating software.

Machine learning libraries offer an attractive alternative.  Due to general applicability, machine learning techniques are widely applied through academia and industry.  A number of mature and widely-used machine learning libraries exist including the scikit-learn \cite{scikit-learn} library for Python and the MLLib library bundled with the Apache Spark distributed-computing engine.  As of the time this was written, scikit-learn and Apache Spark have been around for 8 and 7 years, over 550 and 1,000 contributors, and over 2,600 and 1,100 citations, respectively. Both libraries implement a number of popular machine learning techniques including Random Forests, K-Means clustering, Support Vector Machines (SVMs), and Naive Bayes classifiers. 

The machine learning community is also adept at scaling computationally-complex algorithms to large data sets; scikit-learn and MLLib offer best-in-class performance and scalability.  Scikit-learn is optimized via C and Fortran implementations of performance-critical sections and can efficiently scale through support for parallel processing.  Through Apache Spark and conscientious design, MLLib can scale from a single machine to a large cluster to support data sets too large to fit into memory and algorithms too slow to use productively when running on a single CPU core. Users of scikit-learn and MLLib reap the resulting performance and scalability benefits essentially for free; the libraries largely hide the details from the user.

By adopting machine learning techniques as building blocks for developing bioinformatics methods and applications, machine learning libraries could be adopted, benefitting method developers. Software defects tend to be over-represented in functions employing complex algorithms, parallelization, and low-level optimizations like manual memory management, exactly the functionality that would be replaced by the machine learning libraries, presumably decreasing defect rates and time and labor spent debugging. Additionally, scaling software to large data sets and achieving optimal hardware utilization could be accomplished with far less effort; the optimizations and parallelization techniques employed by machine learning libraries are hidden completely from the user or accomplished by through abstractions that are easier to use.

\section{Contributions}
Prerequisite is to demonstrate that machine learning techniques are applicable to problems in bioinformatics and just as, if not more, effective when replacing or augmenting specialized techniques.  In this dissertation, we consider several of the unique challenges facing bioinformatics of insect genomes and apply machine learning to two common problems in bioinformatics, finding members of a gene family in genomes and ranking genetic variants based on association with population structures.  We demonstrate that both problems are instances of classic machine learning problems, classification and feature selection, respectively.

First, we describe analyzes of the genomes and RNASeq expression of genes from two sand fly species, \emph{Lutzomyia longipalpis} and \emph{Phlebotomus papatasi}, which are vectors of leishmaniasis.  We utilized four analyses to compare the genomes: genome assembly completeness, conservation of gene order (synteny), distributions of ratios of non-silent to silent mutations ($d_N$/$d_S$) in single-copy orthologs, and differential expression of genes in \emph{L. longipalpis} females under sugar-fed, blood-fed and infected-blood fed condition with RNASeq.  Our analysis identified differential expression and synteny blocks of peritrophins, chitinases, and Niemann-Pick type C-2 (NPC2) genes associated with vector-parasite interactions. Peritrophin matrices enable the \emph{Leishmania} parasites in the bloodmeal to survive the digestive process.  Chitinases breakdown the peritrophix matrix, allowing the parasites to escape and remain within in the vector.  NPC2 genes are involves in sterol and sex hormone homeostatis and have been linked to a diverse range of functions including the insect's immune system. These analyses provide examples of the types of questions that biologists seek to answer and the challenges associated with insect genome bioinformatics.  

Secondly, we describe a novel ensemble approach for identifying G Protein-Coupled Receptors (GPCRs) in insect vector genomes.  As part of a genome analysis, genes are annotated and their functionality predicted based on known orthologs. GPCRs possess low levels of sequence conservation, spawning research into methods that can provide more accurate identification and classification than BLAST alone.  Existing GPCR classifiers such as GPCRHMM and PredCouple present several limitations: the classifiers have been trained using GPCRs from a diverse set of model organisms including human (\emph{Homo homo sapiens}) and the fruit fly \emph{Drosophila melanogaster}, resulting in sub-optimal performance when applied to evolutionarily-distance, non-model insect vectors and their scoring systems can be difficult to interpret.  We describe a novel pipeline, called Ensemble*, that combines GPCRHMM, Pfam GPCR Hidden Markov Models (HMMs), and empirical conditional probability functions to improve sensitivity, accuracy, and interpretability over GPCRHMM and Pfam HMMs alone. We evaluated the sensitivity and accuracy of GPCRHMM, PredCouple, the Pfam GPCRH HMMs, and Ensemble*, demonstrating improved sensitivities of up to 10\% on \emph{Aedes aegypti}, \emph{Anopheles gambiae}, \emph{Apis mellifera}, and \emph{Pediculus humanus} with no loss of sensitivity for \emph{Drosophila melanogaster} and \emph{Homo homo sapiens}. Using Ensemble*, we re-analyzed the GPCR repoirteres of the vectors \emph{Aedes aegypti}, \emph{Anopheles gambiae}, resulting in the identification of 30 additional putative GPCRs, of which 19 were validated and confirmed via sequence similarity to known GPCRs with BLAST.

Lastly, we describe a workflow for ranking SNPs from incipient \emph{Anopheles} species using Random Forests.  Separated populations of insect vectors are often compared to identify the genetic basis for differences in insecticide resistance, host preferences, and ecological adaptations.  We demonstrated that Random Forests can be used successfully to rank SNPs in terms of their association with a given population structure.  We identified several causes of bias using numerical experiments and describe appropriate solutions that can be used with standard, \emph{unmodified} implementations of decision trees available in popular machine learning libraries. We provided a software package called Asaph that implements the workflow using Python, numpy, scipy, and scikit-learn.  Asaph was applied to SNPs from \emph{Anopheles gambiae}, \emph{Anopheles coluzzii}, and two forms of \emph{Anopeheles funestus}, Folonzo and Kiribina, resulting in the identification of SNPs in genes previously reportedly to be associated with differences in insecticide resistance between the populations.

Our comparative genomic analysis of \emph{P. papatasi} and \emph{L. longipalpis} provides examples of analyses used by bioinformaticians studying insect vectors.  Ensemble* demonstrates that machine learning techniques can be used to augment existing bioinformatics tools to achieve higher sensitivity and accuracy when applied to new data sets, avoiding the need to develop new methods from scratch. And, lastly, by applying Random Forests to identifying biologically-interesting SNPs in a population genetics framework, we demonstrated that machine learning techniques can replace traditional domain-specific techniques, opening the door to using existing machine learning libraries to simplify bioinformatics application development.

\section{Synopsis}
The rest of this dissertation is organized as follows:

\begin{itemize}
\item Chapter 2: Comparative analysis of the genomes of and RNASeq expression of genes from two sand fly species, \emph{L. longipalpis} and \emph{P. papatasi}.
\item Chapter 3: Evaluation of Hidden Markov Model (HMM)-based tools for identifying G Protein-Coupled Receptors (GPCRs) in insect vector genomes and demonstratation of how a simple approach for creating an ensemble of existing classifiers can improve accuracy by at least 5-11\%.
\item Chapter 4: Application of Random Forests to ranking SNPs from incipient species of \emph{Anopheles} mosquitoes.  We identify multiple sources of bias that can arise and provide solutions that utilize ``out-of-the-box'' implementations of decision trees.  We provide a software implementation of our workflow using the popular machine learning library sci-kit learn, which we apply to the study of SNPs from several \emph{Anopheles} species.
\end{itemize}

We conclude by identifying topics for further investigation.
