\section{Introduction}
As the most populous multi-celled organisms on Earth \textcolor{red}{cite}, insects are of great interest due to their impact on agriculture and role as vectors of human diseases.  \emph{Drosophila suzukii} Matsumura, one of $\approx$ 1,500 species of \emph{Drosophila} flies, has proven to be quite a particularly onerous pest of fruit crops with predicted yield losses of up to 80\% due ot the pest.  \emph{D. suzukii} was originally identified in the early 1900's in Japan, Korea, and China where the flies were found on strawberry and cherry fruits. \emph{D. suzukii} has spread worldwide, first spotted in the mainland United States in Santa Cruz, CA in August 2008. With up to 76\% of the United States' combined cherry, strawberry, raspberry, blackberry, and blueberry commerical production occuring in California, \emph{D. suzukii} has the potential to impact a \$2.6 billion industry. \textcolor{red}{cite Walsh, Hauser, Cini}  

Multiple insect species serve as vectors of human disease.  The mosquito \emph{Anopheles gambiae} is the primary carrier of the \emph{Plasmodium falciparum} parasite that causes malaria. Another mosquito, \emph{Aedes aegypti}, is a vector dengue and yellow fever. The human body louse \emph{Pediculus humanus} spreads thypoid fever. \cite{Fournier2002, Foucault2006, Grimmelikhuijzen2007}

two sand fly species, L. Longipalpis and P. papatasi. Sand flies are ...

Despite their immense impact, many insect species have not yet been studied extensively.  With short life spans and large numbers of offspring, insects undergo relatively rapid evolution, resulting in a broad spectrum of differing biological traits. Feeding preferences, tsetse fly, hosts to pathogens. \textcolor{red}{importance of study to identifying pest management strategies, i5k project}.

The availability of insect genomes enables the identification of novel targets such as GPCRs and rational drug design processes which can produce insecticides, repellents, and other products for the control of vectors such as \emph{An. gambiae} \cite{Grimmelikhuijzen2007, Justice2003}.

As divergent, non-model organisms, insects also present a challenge for bioinformatics. Existing computational tools and mathematical frameworks are often designed with model organisms in mind, needing adaptations to be used with the unique challenges present in the study of non-model insect genomes.

In this thesis, we perform bioinformatic analyses of several insect genomes. Throughout the analysis process, we use a combination of existing tools and create new tools where necessary to improve upon the analysis of insect genomes in the face of their unique challenges.  As part of our work, we developing new tools specific to the study of insects.

This thesis is organized as follows: Chapter 2 provides background on the biological and molecular frameworks as well relevant mathematical and algorithmic formalisms. Chapter 3 comparative analysis of synteny and ka/ks. Chapter 4 differential expression. Chapter 5 - gene annotation GPCRs. Chapter 6 chemosensory receptors. Chapter 7 pop gen. Appendix, bigpetstore