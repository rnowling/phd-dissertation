\begin{acknowledge}
  \epigraph{\textit{Where one alone may be overcome, two together can resist. A three-ply cord is not easily broken.}}{Ecclesiastes 4:12, New American Bible (Revised Edition)}

  During the journey towards earning my Ph.D., I learned that a Ph.D. is not the work of one person alone but rather requires the support of an entire community. I am incredibly grateful and appreciative of the support that many have provided, both directly in helping me successfully complete my Ph.D. and helping me along the path towards a Ph.D. in the first place.

  First, I want to thank my family.  My grandmother Dr. Mary Ellen Jacobs, who earned her Ph.D. later in life after a successful first career as a musician, has always encouraged my intellectual pursuits, inspired me to pursue a Ph.D., opened doors to a position at UConn that started my research career, and provided an empthatic ear as someone who has walked the Ph.D. path before me.  My mother Erika Tracy has always enthusiastically supported (and helped finance!) my educational and scholarly pursuits -- I will be forever grateful of her embrace of my work and trust in my ability to set my own path. My grandfather James Jacobs was always ready with sound advice on people and character, which helped me grow as a person.  My fianc\'{e}e Marissa McCabe has stood by my side and been incredibly flexible while I've been working full time and working on my Ph.D. part-time over the last two years.

  I am incredibly grateful to my advisors, Profs. Mary Ann McDowell and Scott Emrich. Mary Ann  has served as an empaphetic mentor and reliable sounding board from my first day at Notre Dame.  When I decided to take a full-time position at Red Hat, Mary Ann immediately offered, without reservation, to take me on as her student so I could finish my Ph.D. and led a coordinated effort of professors in the Computer Science \& Engineering and Biological Sciences departments to work out the details.   Prof. Scott Emrich was incredibly kind to take me on as a late-stage Ph.D. student who was ``rebooting'' his Ph.D. topic while starting a full-time career.  Scott's mentoring has led me to finding tractable problems and establish a new research direction for myself.  Both took a risk when taking me on as a part-time student and placed a significant amount of trust in me.  Thank you to you both for helping me get to my defense.

  I'd like to thank my colleagues.  Jenica L. Abrudan has been a close friend and collaborator throughout my Ph.D.  I am grateful for the great deal of time she has spent patiently explaining biological concepts, offering constructive feedback, encouragement, and support, and being an awesome friend.  I could not have completed my Ph.D. without the support and encouragement of my management team and colleages at Red Hat.  I would like to thank Scott McClellan and Matthew Farrellee for their support and flexibility.  I would particularly like to thank Will Benton and Erik Erlandson for being open to discussing my ideas and offering constructive feedback, their help in improving my presentation skills, and advice on the overall Ph.D. process.

  I'd made quite a few friends during my time at Notre Dame, both inside and outside of the university.  Thank you to Badi' Abdul-Wahid, Kevin Kastner, Haoyun Feng, Yong Hwan Kim, James Sweet, Li Feng, Mallory Smith, Melinda Varga, and Reid Johnson for their friendship and sharing in the crazy ride that is grad school with me.  Bernardo, Theresa, and Ian Contreras opened their home to me, giving me family away from home. I'd like to thank Sironaj Hindawi, Alexa Robertson, and the larger Baha'i Michiana-area community; my friendships and time spent volunteering with them contributed to my overall development as a person.

  My journey to a Ph.D. started before Notre Dame, and I have many people to thank for helping me get here.  Thank you to profs. Larry Klobutcher, Michael Gryk, and Marty Schiller of the University of Connecticut Health Center for offering me summer research opportunities as a high and college student.  Jay Vyas, a Ph.D. student at the University of Connecticut Health Center, mentored me, helping me to grow in my technical skills, and has become a life-long friend; I am also grateful to Jay for referring me to Red Hat and opening those doors for me.  Thank you to Profs. Holger Mauch, Kelly Debure, and Trevor Cickovski from Eckerd College for the amount of time spent mentoring me and providing opportunities above and beyond the already strong undergraduate education they gave me.

  Last but not least, I would like to thank my committee: Profs. Kevin Bowyer, Greg Madey, Trevor Cickovski, and Zain Syed.  I appreciate the time and effort you've spent to help guide me towards the end of my Ph.D. as well as the significant help in ``rebooting'' my Ph.D. and finding creative ways to satisfy the program requirements so that I could finish the Ph.D. while working full-time.
  
\end{acknowledge}
